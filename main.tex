\documentclass{article}
%%%%%%%%%%%%%%%%%%%%%%%%%%%%%%%%%%%%%%%%%%%%%%%%%%%%%%%%%%%%%%%%%%%%%%%%%%%%%%%%
%%%%%%%%%%%%%%%%%%%%%%%%%% Insert packages here %%%%%%%%%%%%%%%%%%%%%%%%%%%%%%%%
%%%%%%%%%%%%%%%%%%%%%%%%%%%%%%%%%%%%%%%%%%%%%%%%%%%%%%%%%%%%%%%%%%%%%%%%%%%%%%%%


\usepackage{amsmath} % Underbrace and alignment
\usepackage{amssymb}
\usepackage[displaymath, mathlines]{lineno} % Line numbers
\usepackage[margin=1in]{geometry} % Set Amazon margin (1 inch)
\usepackage{fancyhdr} % Headers and Footers
\usepackage{lastpage} % for lastpage macro in footer
\usepackage{titling} % For moving title and styling it
\usepackage{hyperref} % for URLs
\usepackage{listings} % for lstlisting environment (code listings)
\usepackage{booktabs} % for fancy tables
\usepackage{graphicx} % for inserting images
\usepackage[numbers]{natbib} % for mananging your references
\usepackage{optidef} % for fancy optimization formulation
\usepackage{algorithm} %for fancy algorithms
\usepackage{algpseudocode} %for fancy algorithms
\usepackage{bm} %for using bold in math mode



%%%%%%%%%%%%%%%%%%%%%%%%%%%%%%%%%%%%%%%%%%%%%%%%%%%%%%%%%%%%%%%%%%%%%%%%%%%%%%%%
%%%%%%%%%%%%%%%%%%%%%%%%%%%%Template configuration%%%%%%%%%%%%%%%%%%%%%%%%%%%%%%
%%%%%%%%%%%%%%%%%%%%%%%%%%%%%%%%%%%%%%%%%%%%%%%%%%%%%%%%%%%%%%%%%%%%%%%%%%%%%%%%

  %\usepackage[hang,flushmargin]{footmisc}  % Disable indent in footer
  \usepackage{longtable} % For multi-page table
  \usepackage{arydshln} % For dashed line
  \usepackage[toc,page]{appendix} % For appendices
   
  % Used for characters in section formatting
  \usepackage[utf8]{inputenc}
  \usepackage[T1]{fontenc}
  \usepackage{soulutf8} 
   
  \usepackage[]{titlesec} % For defining underlined section titles
   
  \setlength{\droptitle}{-3cm} % Move title up
  \posttitle{\par\end{center}} % Remove extra space after title
   
  % Ensure titlesec doesn't overwrite thetitle macro (so we can use it in header)
  \let\oldtitle\title
  \renewcommand{\title}[1]{\oldtitle{#1}\newcommand{\mythetitle}{#1}}
   
  % Increase linenumber font size
  \def\linenumberfont{\normalfont\small\sffamily}
   
  % Colour hyperlinks (and remove squares)
  \usepackage{xcolor}
  \hypersetup{
      colorlinks,
      linkcolor={red!50!black},
      citecolor={blue!50!black},
      urlcolor={blue!80!black}
  }

  % Define header and footer on normal pages
  \pagestyle{fancy}
  \fancyhf{}
  \renewcommand{\headrulewidth}{0.5pt}
  \renewcommand{\footrulewidth}{0.5pt}
  \lfoot[Amazon Confidential]{Amazon Confidential}
  \rfoot[Page \thepage~of \pageref{LastPage}]{Page \thepage~of \pageref{LastPage}}
  \lhead[\mythetitle]{\mythetitle}
  \rhead[\thedate]{\thedate}
   
  % Re-define "plain" title page to include footer but not header
  \fancypagestyle{plain}{
  \fancyhf{}
  \renewcommand{\headrulewidth}{0pt}
  \renewcommand{\footrulewidth}{0.5pt}
  \lfoot[Tutorial]{Tutorial}
  \rfoot[Page \thepage~of \pageref{LastPage}]{Page \thepage~of \pageref{LastPage}}
  }
   
  % Do not indent paragraphs
  \setlength\parindent{0pt}

  % Ensure sections are underlined and bolded - Amazon formatting
  \titleformat{\section}
    {\normalfont\Large\bfseries}{\thesection}{1em}{}[{\titlerule[0.8pt]}] 


  % Declaring your bibliography style
  % \bibliographystyle{humannat}
  \bibliographystyle{abbrvnat}


%%%%%%%%%%%%%%%%%%%%%%%%%%%%%%%%%%%%%%%%%%%%%%%%%%%%%%%%%%%%%%%%%%%%%%%%%%%%%%%%
%%%%%%%%%%%%%%%%%%%%%%%%%% Begin your document %%%%%%%%%%%%%%%%%%%%%%%%%%%%%%%%%
%%%%%%%%%%%%%%%%%%%%%%%%%%%%%%%%%%%%%%%%%%%%%%%%%%%%%%%%%%%%%%%%%%%%%%%%%%%%%%%%


\title{A Tutorial of Heavy-traffic Approximation in Queueing Theory}
\author{}
\date{August 2021}

\begin{document}

\maketitle

\section{Introduction}
The purpose of this document is to introduce the concept of heavy-traffic approximation in Queueing Theory to science community in XXX. A system of large scale is usually difficult to analyze due to analytical intractability. For example, in call center, a simple model is the Erlang-A, aka, $M/M/N+M$, where the first $M$ stands for Poisson arrivals, the second $M$ for exponential service times, $N\geq 1$ for the number of servers and $+M$ for the exponential patience time. To calculate the steady-state distribution for number of contacts in the system, it involves the calculation of $N!$. See Section \ref{sec:appendix-erlanga} for more detail. When $N$ is large, which is often the case given the scale of operations in XXX, the computation of $N!$ can be very expensive. However, heavy-traffic theorems take advantage of large system scale and produce simple yet useful results. 

In this tutorial, we properly define the concept of system \textit{scale} and \textit{heavy-traffic}. (I also want to review how the scale affects system behavior.) We will introduce two types of heavy-traffic results and their application in queueing theory: (1) conventional heavy-traffic limits and (2) many-server heavy-traffic.  But before we dive deep into the theoretical results, we first answer the question: how good is the heavy-traffic approximation, comparing to exact analysis? 

\section{Quality of Approximation}

\section{Conventional Heavy-traffic Limits}

\section{Many-server Heavy-traffic Limits}

\section{Appendix}
We provide additional technical nodes in this Appendix. 

\subsection{Exact analysis of Erlang-A model}\label{sec:appendix-erlanga}
The Erlang-A model assumes exponential inter-arrival time with rate $\lambda$, exponential service(handle) rate $\mu$, $N$ parallel agents, and exponential patience time with mean $1/\gamma$. One can construct a simple Markov Chain and write the transition rate matrix to obtain the steady-state distribution ($\mathbf{\pi}$) for number of contact in the system, denoted as $X$. For given $N$ number of agents, let $\pi_n\equiv\mathbf{P}(X=n)$ denote the steady-state distribution
\begin{align}
	\pi_n=\begin{cases}
		 \dfrac{1}{n!}\left({\dfrac{\lambda}{\mu}}\right)^n\pi_0,\quad n=0,1,\cdots,N\\
		 \dfrac{1}{N!}
		 {\left(\dfrac{\lambda}{\mu}\right)}^n
		 \left(
		 \prod_{j=N+1}^n\dfrac{\lambda}{N\mu+(j-N)\gamma}
		 \right)\pi_0, \quad n=N+1, \cdots
	\end{cases}
\end{align}where 
\begin{align}
	\pi_0=\left[1+\sum_{j=1}^{N}\dfrac{1}{j!}\left({\dfrac{\lambda}{\mu}}\right)^n
	+\dfrac{1}{N}\left({\dfrac{\lambda}{\mu}}\right)^N
	\left(\sum_{i=N+1}^{\infty}\prod_{j=N+1}^i\dfrac{\lambda}{N\mu+(j-N)\gamma}\right)
	\right]^{-1}.
\end{align}



\end{document}
